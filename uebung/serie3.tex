\documentclass[12pt,a4paper]{scrartcl}
\usepackage{ifpdf}
\usepackage[utf8]{inputenc}
\usepackage[ngerman]{babel}
\usepackage{amsmath}
\usepackage{amssymb}
\usepackage{fancyhdr}
\usepackage{listings}
\usepackage{color}
\usepackage{pdfpages}
\usepackage{stmaryrd}
\lstset{% general command to set parameter(s) 
basicstyle=\small	% print whole listing small
}

\renewcommand{\author}[1]{\newcommand{\printAuthor}{#1}}
\renewcommand{\title}[1]{\newcommand{\printTitle}{#1}}
\title{Serie 01}
\author{Johannes Hedtrich, Nikolaus Majewski}
\date{\today}

\pagestyle{fancy}
\fancyhead{}
\fancyfoot{}
\fancyhead[LO,LE]{\printAuthor}
\fancyhead[RO,RE]{\printTitle}
\fancyfoot[CO,CE]{\thepage}
 
\title{Serie 03}

\begin{document} 
\section*{Aufgabe 3.3 Knapsack}

\section*{Aufgabe 3.4 Greedy-Algorithmen}

\subsection*{(a)}

\begin{lstlisting}[language=Ruby,numbers=right]
class CoinDistribution
  attr_accessor :coins
  
  def initialize(*available_coins)
    self.coins = available_coins
    self.coins.sort! {|a,b| b <=> a}
  end
  
  def to_coins(amount)
    distribution = Hash.new
    self.coins.each do |coin|
      count = amount / coin
      amount -= count * coin
      distribution[coin] = count
      break if amount == 0
    end
    distribution
  end
end  
\end{lstlisting}

\subsection*{(b)}

Sei $W = {1,5,10,25}$.\\
Für jede Münze $m \in W$ gilt, dass ihr Wert maximal dem halben Wert der nächst größeren Münze entspricht.\\ 


\subsection*{(c)}

Für $W = {1,6,9}$ und $g = 14$ liefert der Algorithmus nicht das "optimale" Ergebnis, sondern: $5 * 1$ und $1 * 9$ anstatt: $2 * 6$ und $2 * 1$.

\end{document}
