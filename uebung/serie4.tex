\documentclass[12pt,a4paper]{scrartcl}
\usepackage{ifpdf}
\usepackage[utf8]{inputenc}
\usepackage[ngerman]{babel}
\usepackage{amsmath}
\usepackage{amssymb}
\usepackage{fancyhdr}
\usepackage{listings}
\usepackage{color}
\usepackage{pdfpages}
\usepackage{stmaryrd}
\usepackage{algorithmic}

\lstset{% general command to set parameter(s) 
basicstyle=\small	% print whole listing small
}

\renewcommand{\author}[1]{\newcommand{\printAuthor}{#1}}
\renewcommand{\title}[1]{\newcommand{\printTitle}{#1}}
\author{Johannes Hedtrich, Nikolaus Majewski}
\date{\today}

\pagestyle{fancy}
\fancyhead{}
\fancyfoot{}
\fancyhead[LO,LE]{\printAuthor}
\fancyhead[RO,RE]{\printTitle}
\fancyfoot[CO,CE]{\thepage}

\newcommand{\uuml}{\"{u}}
\newcommand{\auml}{\"{a}}
\newcommand{\ouml}{\"{o}}

\newcommand{\Uuml}{\"{U}}
\newcommand{\Auml}{\"{A}}
\newcommand{\Ouml}{\"{O}}

 
\title{Serie 04}

\begin{document} 

\section*{Aufgabe 4.3 Starke Zusammenhangskomponenten}

\textbf{Voraussetzung: }Sei $D=(V,E)$ ein Digraph.\\
\noindent
\textbf{Behauptung: } Es sei r der erste Knoten, für den im Algorithmus zur Ermittlung der starken Zusammenhangskomponenten in Schritt (7) die Bedingung $T(r) =  DFS(r)$ gilt. Dann liegen neben den Knoten auf dem Stack oberhalb von r keine weiteren Knoten in der starken Zusammenhangskomponente zu r.\\
\noindent
\textbf{Beweis: }

Annahme: Es existiert eine ausgehende Kante $e = (v,u)$ mit $DFS(v) \geq DFS(r)$, $DFS(u) < DFS(r)$ und es existiert ein gerichteter Weg von $u$ nach $r$. Somit würden diese Knoten zur starken Zusammenhangskomponente von r gehören. Jedoch wäre dann im Backtracking $T(w) \leq DFS(u) < DFS(r) = T(r)$ mit $w \in \{u --> r\}$ als minimale Tiefe zurück gereicht worden, und somit wäre in r $T(r) != DFS(r)$.


%\section*{Aufgabe 4.4 Tiefensuche in Digraphen}

\end{document}
