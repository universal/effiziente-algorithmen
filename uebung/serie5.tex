\documentclass[12pt,a4paper]{scrartcl}
\usepackage{ifpdf}
\usepackage[utf8]{inputenc}
\usepackage[ngerman]{babel}
\usepackage{amsmath}
\usepackage{amssymb}
\usepackage{fancyhdr}
\usepackage{listings}
\usepackage{color}
\usepackage{pdfpages}
\usepackage{stmaryrd}
\lstset{% general command to set parameter(s) 
basicstyle=\small	% print whole listing small
}

\renewcommand{\author}[1]{\newcommand{\printAuthor}{#1}}
\renewcommand{\title}[1]{\newcommand{\printTitle}{#1}}
\title{Serie 01}
\author{Johannes Hedtrich, Nikolaus Majewski}
\date{\today}

\pagestyle{fancy}
\fancyhead{}
\fancyfoot{}
\fancyhead[LO,LE]{\printAuthor}
\fancyhead[RO,RE]{\printTitle}
\fancyfoot[CO,CE]{\thepage}
 
\title{Serie 05}

\begin{document} 

\section*{Aufgabe 5.3 Matroide II}

\textbf{Voraussetzung: }

Sei $U=\{x_1, \cdots, x_{|U|}, \mathfrak{I}\}$ ein Matroid und $w: U \rightarrow \mathbb{N}$ eine beliebige Gewichtsfunktion. Sei $W(X) = \sum\{w(x) : x \in X\}$. Der Algorithmus sei wie in der Aufgabe definiert.

\noindent
\textbf{Behauptung: } 

Der Algorithmus findet eine unabhängige Menge $S$ mit maximalen Gesamtgewicht $W(S)$.

\noindent
\textbf{Beweis: }
Sei $s_i$ mit $s_i \in S$ das im i-ten Schritt vom Algorithmus gewählte Element.
Annahme: Es existiert eine Menge $T = \{x_{t_1}, \cdots x_{t_{|T|}}\}$ mit $W(T)$ optimal und $W(T) > W(S)$. 
Die Lösung des Algorithmus und die optimale Lösung T sind inklusionsmaximal, also $|S| = |T|$. Sei o.B.d.A $x_{t_k} \geq x_{t_{k+1}}$ mit $ 1 \leq k \leq |T|$. Es existiert also ein $t_j \in T$ mit $w(t_j) > w(s_j), s \in S$. Jedoch hätte der Algorithmus im j-ten Schritt eben dieses Element $t_j$ statt $s_j$ gewählt, da $S' \cup t_j$ mit $S' = \{s_i \in S | i < j\}$ nach Definition auch in $\mathfrak{I}$ enthalten ist. $\lightning$

%\section*{Aufgabe 4.4 Tiefensuche in Digraphen}

\end{document}
