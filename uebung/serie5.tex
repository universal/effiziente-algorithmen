\documentclass[12pt,a4paper]{scrartcl}
\usepackage{ifpdf}
\usepackage[utf8]{inputenc}
\usepackage[ngerman]{babel}
\usepackage{amsmath}
\usepackage{amssymb}
\usepackage{fancyhdr}
\usepackage{listings}
\usepackage{color}
\usepackage{pdfpages}
\usepackage{stmaryrd}
\usepackage{algorithmic}

\lstset{% general command to set parameter(s) 
basicstyle=\small	% print whole listing small
}

\renewcommand{\author}[1]{\newcommand{\printAuthor}{#1}}
\renewcommand{\title}[1]{\newcommand{\printTitle}{#1}}
\author{Johannes Hedtrich, Nikolaus Majewski}
\date{\today}

\pagestyle{fancy}
\fancyhead{}
\fancyfoot{}
\fancyhead[LO,LE]{\printAuthor}
\fancyhead[RO,RE]{\printTitle}
\fancyfoot[CO,CE]{\thepage}

\newcommand{\uuml}{\"{u}}
\newcommand{\auml}{\"{a}}
\newcommand{\ouml}{\"{o}}

\newcommand{\Uuml}{\"{U}}
\newcommand{\Auml}{\"{A}}
\newcommand{\Ouml}{\"{O}}

 
\title{Serie 05}

\begin{document} 

\section*{Aufgabe 5.3 Matroide II}

\textbf{Voraussetzung: }

Sei $U=\{x_1, \cdots, x_{|U|}, \mathfrak{I}\}$ ein Matroid und $w: U \rightarrow \mathbb{N}$ eine beliebige Gewichtsfunktion. Sei $W(X) = \sum\{w(x) : x \in X\}$. Der Algorithmus sei wie in der Aufgabe definiert.

\noindent
\textbf{Behauptung: } 

Der Algorithmus findet eine unabhängige Menge $S$ mit maximalen Gesamtgewicht $W(S)$.

\noindent
\textbf{Beweis: }
Sei $s_i$ mit $s_i \in S$ das im i-ten Schritt vom Algorithmus gewählte Element.
Annahme: Es existiert eine Menge $T = \{x_{t_1}, \cdots x_{t_{|T|}}\}$ mit $W(T)$ optimal und $W(T) > W(S)$. 
Die Lösung des Algorithmus und die optimale Lösung T sind inklusionsmaximal, also $|S| = |T|$. Sei o.B.d.A $x_{t_k} \geq x_{t_{k+1}}$ mit $ 1 \leq k \leq |T|$. Es existiert also ein $t_j \in T$ mit $w(t_j) > w(s_j), s \in S$. Jedoch hätte der Algorithmus im j-ten Schritt eben dieses Element $t_j$ statt $s_j$ gewählt, da $S' \cup t_j$ mit $S' = \{s_i \in S | i < j\}$ nach Definition auch in $\mathfrak{I}$ enthalten ist. $\lightning$

\clearpage

\section*{Aufgabe 5.4 Transitive Reduktion}
\textbf{Voraussetzung: }

Sei $D = (V,E)$ ein gerichteter, endlicher, azyklischer, schleifenfreier Digraph. 

\noindent
\textbf{Behauptung: } 

Für azyklische Graphen ist die transitive Reduktion eindeutig bestimmt durch:\\ 
$E^{'} = \bigcap \{ E^{''} \subseteq E : E^{''^*} = E^{^*} \}$

\noindent
\textbf{Beweis: }

(1) Es existiert ein Pfad von $u$ nach $v$ in $D$ genau dann, wenn ein solcher in $D^{'}$ existiert, also $E^{^*} = E^{'^*}$.

Aus $E^{''^*} = E^{^*}$ folgt das für jedes Paar $(u,v)$ für das ein Pfad in $E$ existiert, auch einer in $E^{''}$ vorhanden ist. Dieser Pfad kann auf 3 Möglichkeiten entstanden sein: 

\begin{itemize}

	\item Der einzige Pfad ist die direkte Kante $(u,v)$. Dann ist diese offensichtlich in allen $E^{''}$ enthalten, und somit auch im Schnitt über diese.
	\item Es existiert keine direkte Kante $(u,v)$, sondern lediglich ein Pfad von $u$ nach $v$ über Zwischenknoten $k_i$. Dann gilt für die Paare $(u, k_1), \cdots, (k_j, v)$ das für diese in allen $E^{''}$ ein Pfad enthalten sein muss, der rekursiv wiederum eine der 3 Möglichkeiten erfüllt.
	\item Es existiert sowohl die direkte Kante $(u,v)$ als auch ein Pfad von $u$ nach $v$ über Zwischenknoten $k_i$. Dann gilt für die Paare $(u, k_1), \cdots, (k_j, v)$ das für diese in allen $E^{''}$ ein Pfad enthalten sein muss, der rekursiv wiederum eine der 3 Möglichkeiten erfüllt, lediglich die direkte Kante $(u,v)$ muss nicht in $E^{''}$ enthalten sein.
\end{itemize}

(2) $|E^{'}|$ ist minimal.

Annahme: $|E^{'}|$ ist nicht minimal. Dann existiert eine Kante $k \in E^{'}$ sodass $(E^{'}\backslash k)^{^*} = E^{^*}$. Da $E^{'} \subseteq E^{''} : E^{''^*} = E^{^*}$, $k \in E^{''}$. Zudem gilt auch $(E^{'}\backslash k) \subseteq (E^{''}\backslash k)$, somit gilt auch $(E^{''}\backslash k)^{^*} = E^{^*}$. Dann wäre jedoch $E^{''}\backslash k$ in der Schnittmenge enthalten, und somit $k$ nicht im Resultat. $\lightning$


\end{document}
