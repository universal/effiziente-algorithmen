\documentclass[12pt,a4paper]{scrartcl}
\usepackage{ifpdf}
\usepackage[utf8]{inputenc}
\usepackage[ngerman]{babel}
\usepackage{amsmath}
\usepackage{amssymb}
\usepackage{fancyhdr}
\usepackage{listings}
\usepackage{color}
\usepackage{pdfpages}
\usepackage{stmaryrd}
\lstset{% general command to set parameter(s) 
basicstyle=\small	% print whole listing small
}

\renewcommand{\author}[1]{\newcommand{\printAuthor}{#1}}
\renewcommand{\title}[1]{\newcommand{\printTitle}{#1}}
\title{Serie 01}
\author{Johannes Hedtrich, Nikolaus Majewski}
\date{\today}

\pagestyle{fancy}
\fancyhead{}
\fancyfoot{}
\fancyhead[LO,LE]{\printAuthor}
\fancyhead[RO,RE]{\printTitle}
\fancyfoot[CO,CE]{\thepage}
 
\title{Serie 06}

\begin{document} 

  \section*{Aufgabe 6.3 Eulerkreise II}
  
  \textbf{Voraussetzung: } Sei $G=(V,E)$ ein Multigraph.
  
  \noindent
  \textbf{Behauptung: } Der Algorithmus zur Bestimmung eines Eulerkreises läuft in $O(|V| + |E|)$.

  \noindent
  \textbf{Beweis: }
  
  Wähle folgende Datenstrukturen:
  Adjazenzliste: doppelt verkettete Liste mit Speicherung der Größe
  Kreis: einfach verkettete Liste mit Pointer auf Start-, End- und Zwischenknoten $w_i$
  
  \begin{enumerate}
    \item Test ob $G$ zusammenhängend ist lässt sich mit DFS lösen $\Rightarrow O(|V| + |E|)$.\\
          Test auf geraden Grad: Wenn man zu jedem Knoten die Adjazenzliste mit deren Größe speichert, ist der Test in $O(|V|)$ möglich. Insgesamt also Schritt(1) in $O(|V| + |E|)$.
    \item Bestimmung eines Kreises $K_i = (l_1, \cdots, l_k, l_1)$: Speichert man die Adjazenzliste als verkettete Liste, die nach aufsteigenden Knotenindizes sortiert ist, so ist das Auswählen einer unbesuchten Kante in $O(1)$ möglich. Ebenso das markieren der Kante als besucht, hierzu wird die Kante einfach entfernt, was wiederum in $O(1)$ möglich ist. In diesem Schritt werden genau $k$ Kanten betrachtet. Da der Algorithmus nur unbesuchte Kanten betrachtet, werden über alle ''Aufrufe'' von Schritt(2) genau $\sum_{K_i} |E_{K_i}| = |E|$ Kanten betrachtet, somit ist dieser Schritt insgesamt in $O(E)$.
    \item Speichert man in Schritt(2) die Anzahl der besuchten Kanten, so ist dieser Schritt in $O(1)$ möglich.
    \item Zur Bestimmung des nächsten Startknotens $w$ läuft man auf dem bisher gefunden Kreis entlang, und setzt den Pointer für $w$ entsprechend mit um. Dies geschieht solange, bis man einen Knoten findet, der noch unbesuchte Kanten besitzt. Die Überprüfung ob ein Knoten unbesuchte Kanten besitzt, lässt sich in $O(1)$ lösen. Da die später gefundenen Kreise nie vor $w$ auf dem bisher gefunden Kreis enden können, sind diese Schritte auch durch $|E|$ beschränkt. Somit laufen alle ''Aufrufe'' dieses Schritts ingesamt in $O(|E|)$
    \item Da der Start- und Endknoten des hinzuzufügenden Kreises gerade der Zwischenknoten $w$ ist, ist das zusammenfügen von 2 Teilkreisen in $O(1)$ möglich, hierzu entfernt man einfach aus dem hinzuzufügenden Kreis den Startknoten und setzt den Pointer aufs nächste Element von $w$ auf den folgenden Knoten.
  \end{enumerate}
  
  Da sich alle Teilschritte über alle ''Aufrufe'' in $O(|V|)$, $O(|E|)$ oder $O(|V| + |E|)$, ist auch der gesamte Algorithmus in $O(|V| + |E|)$.
\clearpage
  \section*{Aufgabe 6.4 Eulerkreise III}
  
  \textbf{Voraussetzung: } Sei $G=(V,E)$ ein Graph und $L(G) = (E, L(E))$ der durch $\{e_1,e_2\} \in L(E) \Leftrightarrow e_1,e_2 \in E, |e_1 \cap e_2| = 1$ definierte Linegraph. Sei zudem $G$ eulersch.
  
  \noindent
  \textbf{Behauptung: } $G$ eulersch $\Rightarrow$ $L(G)$ ist eulersch und hat einen Hamiltonkreis.
  
  \noindent
  \textbf{Beweis:}
  
  $G$ eulersch $\Rightarrow L(G)$ hat einen Hamiltonkreis.\\
  
  Sei $K_G$ ein zu $G$ gehörender Eulerkreis. Da die Knoten in $L(G)$ gerade den Kanten in $G$ und die Kanten gerade den möglichen Wegstücken $v_1,v_2,v_3$ in $G$ entsprechen, so ergibt sich aus $K_G$ direkt der Hamiltonkreis in $L(G)$.
  
  $G$ eulersch $\Rightarrow$ $L(G)$ ist eulersch.\\

  Zerlegt man $G$ in die kleinstmöglichen Kreise 
  $K_i = (l_1, \cdots, l_k) | l_1 = l_k \wedge \forall l_i \not \exists l_j l_i = l_j | 1 < i,j < k$,
   so ist die Vereinigung dieser Kreise ein Eulerkreis. Für einen Kreis $K_i$ und dem zugehörigen Graphen $G_{K_i}$ ist $L(G_{K_i})$ offensichtlich wiederum eulersch. Somit bleibt noch zu betrachten was an der Verbindungsstelle zweier solcher Kreise $K_1, K_2$ passiert:  Sei $k$ der Schnittknoten, $k_1, k_2$ die zu $K_1$ und $k_3, k_4$ die zu $K_2$ gehörenden Knoten. Somit sind die zu betrachtenden Kanten $e_1 = (k_1, k), e_2 = (k_2, k), e_3 = (k, k_3), e_4 = (k, k_4)$. So ergibt sich im zugehörigen folgende Kantenmenge $l_1 = (e_1,e_2), l_2 = (e_1, e_3), l_3 = (e_1, e_4), l_4 = (e_2, e_3), l_5 = (e_2,e_4), l_6 = (e_3, e_4)$. Hiervon gehören die Kanten $l_1$ und $l_6$ schon zu eulerschen Teilkreisen in $L(G)$. Die restlichen Kanten bilden auch einen Eulerkreis, nämlich: $l_2, l_4, l_5, l_3$. Da die Vereinigung von Eulerkreisen wiederum eulersch ist, ist $L(G)$ eulersch.
  
  
  




\end{document}