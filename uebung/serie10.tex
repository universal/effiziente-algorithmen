\documentclass[12pt,a4paper]{scrartcl}
\usepackage{ifpdf}
\usepackage[utf8]{inputenc}
\usepackage[ngerman]{babel}
\usepackage{amsmath}
\usepackage{amssymb}
\usepackage{fancyhdr}
\usepackage{listings}
\usepackage{color}
\usepackage{pdfpages}
\usepackage{stmaryrd}
\lstset{% general command to set parameter(s) 
basicstyle=\small	% print whole listing small
}

\renewcommand{\author}[1]{\newcommand{\printAuthor}{#1}}
\renewcommand{\title}[1]{\newcommand{\printTitle}{#1}}
\title{Serie 01}
\author{Johannes Hedtrich, Nikolaus Majewski}
\date{\today}

\pagestyle{fancy}
\fancyhead{}
\fancyfoot{}
\fancyhead[LO,LE]{\printAuthor}
\fancyhead[RO,RE]{\printTitle}
\fancyfoot[CO,CE]{\thepage}
 
\title{Serie 10}

\begin{document} 

  \section*{Aufgabe 10.3 (WEIGHTED MAXCUT)}
  \textbf{Voraussetzung: } Sei $G = (V,E)$ ein Graph. Sei $w(e) \in \mathbb{Q}_+$ Kantengewicht.
  
  \noindent
  \textbf{gesucht: } eine Partition $(S, V\\S)$ der Knotenmenge mit maximaler Summe von Kantengewichten im Schnitt.
  
  \noindent
  \textbf{Alg.: }
  
  $S := \emptyset$\\
  while( $\exists v \in V$ so dass das verschieben von v von einer Seite zur anderen Seite im Schnitt dessen Wert erhöht)\\
  do (verschiebe einen solchen Knoten $v \in V$ zur anderen Seite)\\
  end

  \noindent
  \textbf{Behauptung: } Der Algorithmus liefert zu jedem Graphen G und Kantengewichten w einen Schnitt, für dessen Größe k gilt $k \geq \frac{1}{2} OPT$.
  
  \noindent
  \textbf{Beweis: }
  
  Sei $m(v) = \sum_{ e = (v, v_1) || (v_1, v)} w(e)$ die Summe der der Gewichte der Kanten eines Knotens.
  Es sei $(S, V\\S)$ der vom Algorithmus berechnete Schnitt. Dann gilt f.a $v \in V$, dass mindestens $\frac{1}{2} m(v)$ im Schnitt liegt. Somit ist $k \geq \frac{1}{2} \sum_{e \in E} w(e)$. Es gilt $OPT \leq \sum_{e \in E} w(e)$. Somit gilt $k \geq \frac{1}{2} OPT$.
  
  \section*{Aufgabe 10.4 (BIN PACKING)}
  \textbf{Voraussetzung: } Seien $a_1, \cdots , a_n \in ]0,1[$ Objektgrößen.
  
  \noindent
  \textbf{gesucht: } eine Packung der Objekte in möglichst wenig Bins der Kapazität 1.
  
  \noindent
  \textbf{Algorithmus: } 
  Füge die Objekte nacheinander in den jeweils ersten passenden Bin, falls kein solches existiert, füge es in ein neues Bin ein.
  
  \noindent
  \textbf{Behauptung: } Der Algorithmus benötigt höchstens $3OPT$ Bins. 
  
  \noindent
  \textbf{Beweis: }
    
  Zu jedem Zeitpunkt während des Befüllens der Bins ist immer höchstens einer nicht ganz halb voll, denn der Algorithmus würde für ein Objekt der Größe $<= \frac{1}{2}$ kein neues Bin aufmachen, sondern lediglich möglicherweise für ein Objekt der Größe $> \frac{1}{2}$. 
  
  Hat man nun $k$ Bins, sind wenigstens $k - 1$ mehr als halb voll, also $\sum_{i}^{n} a_i > \frac{k - 1}{2}$. Da $\sum_{i}^{n} a_i \leq OPT$ gilt: $2OPT > k-1$, also $k < 2OPT + 1 <= 3OPT$.
\end{document}