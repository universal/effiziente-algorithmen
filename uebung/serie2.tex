\documentclass[12pt,a4paper]{scrartcl}
\usepackage{ifpdf}
\usepackage[utf8]{inputenc}
\usepackage[ngerman]{babel}
\usepackage{amsmath}
\usepackage{amssymb}
\usepackage{fancyhdr}
\usepackage{listings}
\usepackage{color}
\usepackage{pdfpages}
\usepackage{stmaryrd}
\lstset{% general command to set parameter(s) 
basicstyle=\small	% print whole listing small
}

\renewcommand{\author}[1]{\newcommand{\printAuthor}{#1}}
\renewcommand{\title}[1]{\newcommand{\printTitle}{#1}}
\title{Serie 01}
\author{Johannes Hedtrich, Nikolaus Majewski}
\date{\today}

\pagestyle{fancy}
\fancyhead{}
\fancyfoot{}
\fancyhead[LO,LE]{\printAuthor}
\fancyhead[RO,RE]{\printTitle}
\fancyfoot[CO,CE]{\thepage}

\title{Serie 02}

\begin{document}
\section*{Aufgabe 2.3}

  	$ T(n) = a T(\frac{n}{c}) + b n^{k} \hspace{1cm},k \in\mathbb{N} $
  	\newline
  	Aufl\ouml sen ergibt:\\
  \begin{eqnarray}
  T(n) &=& = aT(\frac{n}{c}) + bn^{k}\\
  &=& a ( aT(\frac{n}{c^{2}}) + b\frac{n^{k}}{c} ) + b n^{k}\\
  &=& a (a ( aT(\frac{n}{c^{3}}) + b \frac{n^{k}}{c^{2}} ) + b \frac{n^{k}}{c} ) + b n^{k}\\
  &=& ...\\
  &=& bn^{k} + b \frac{n^{k}}{c}a + b \frac{n^{k}}{c^{2}}a^{2} + b \frac{n^{k}}{c^{3}}a^{3} + ...\\
  &=& bn^{k} \sum\limits_{i=0}^{\log_{c}(n) - 1}{({\frac{a}{c}})}^{i}\\
  \end{eqnarray}

  Dadurch ergeben sich 3 F\auml lle die zu betrachten sind:\\

  Fall 2: $a = c^{k}$
  \begin{eqnarray}
  T(n) &=& bn^{k}(log_{c}(n) - 1 + 1)\\
  &=& bn^{k}(log_{c}(n))\\
  &=& O (n^{k}(log_{c}(n))\\
  \end{eqnarray}


  Fall 2: $a < c^{k}$
  \begin{eqnarray}
  T(n) &=& bn^{k} \sum\limits_{i=0}^{\log_{c}(n) - 1}{({\frac{a}{c}})}^{i}\\
  &<& bn^{k} \sum\limits_{i=0}^{\infty}{({\frac{a}{c}})}^{i} = \frac{bn^{k}}{1- \frac{a}{c}} = bn^{k} \frac{1}{1- \frac{a}{c}}\\
  &=& O (n^{k}) \hspace{1cm}, da \quad\frac{1}{1- \frac{a}{c}} \ konst.\\
  \end{eqnarray}


  Fall 3: $a > c^{k}$
  \begin{eqnarray}
  T(n) &=& bn^{k} \sum\limits_{i=0}^{\log_{c}(n) - 1}{({\frac{a}{c}})}^{i} = bn^{k} \frac{{(\frac{a}{c})}^{log_{c}(n)- 1 +1} - 1}{\frac{a}{c} - 1}\\
  &=& bn^{k} \frac{{(\frac{a}{c})}^{log_{c}(n)} - 1}{\frac{a}{c} - 1}\\
  &\leq& \frac{bn^{k}}{\frac{a}{c}-1} \frac{a^{log_{c}(n)}}{c^{log_{c}(n)}} = \frac{bn^{k}}{\frac{a}{c}-1} \frac{a^{log_{c}(n)}}{c}\\
  &=& \frac{bn^{k} a^{log_{c}(n)}}{c(\frac{a}{c} - 1)} = \frac{bn^{k} a^{log_{c}(n)}}{a- \frac{1}{c}}\\
  &=& \frac{bn^{k} a^{log_{c}(n)}}{a - \frac{1}{c}} = \frac{b}{a - \frac{1}{c}} n^{k}a^{log_{c}(n)}\\
  &=& O(n^{k}a^{log_{c}(n)}) \hspace{1cm}, da \frac{b}{a - \frac{1}{c}} \ konst.\\
  &=& O(n^{k}n^{log_{c}(a)}) = O(n^{log_{c}(a)+ k} )\\ 
  \end{eqnarray}
  
  

\section*{Aufgabe 2.4}

Umformen/Aufl\ouml sen ergibt:\\
\begin{eqnarray*}
\sum\limits_{i=0}^{n}{c_i x^i} &=& c_0 + \sum\limits_{i=1}^{n}{c_i x^i}\\
&=& c_0 + c_1 x^1 + c_2 x^2 + c_3 x^3 + ... + c_n x^n\\
&=& c_0 + c_1 (x) + c_2 (x * x) + c_3 (x * x * x) + ... + c_n (x * x * x * x * ... *n)\\
\end{eqnarray*}
Dadurch ergibt sich bei Betrachtung der Anzahl der Multiplikationen:\\
$M$ = Multiplikation
\begin{eqnarray*}
&=& 0M + 1M + 2M + 3M + ... + nM\\
&=& \frac{n (n - 1)}{2}M\\
\end{eqnarray*}
Offensichtlich gilt fuer die Anzahl der Additionen:\\
$A$ = Addition
\begin{equation*}
= nA
\end{equation*}

Gesamt im Einheitskostenma\ss:\\
\begin{equation*}
= n + \frac{n (n - 1)}{2} = n + n + \frac{n^2}{2} - \frac{n}{2} = n (1 + \frac{n}{2} - \frac{1}{2}) = n(\frac{1}{2}+ \frac{n}{2})\\
\end{equation*}

\newpage

Und nun der 2te Term:
\begin{eqnarray*}
&=& (... ((c_{n}x + c_{n - 1})x + c_{n - 2}) ...)x + c_0\\
&=& (... ((((c_{n}x + c_{n - 1})x + c_{n - 2})x + c_{n - 3})x + c_{n - 4})x + c_{n - 5}) ...)x + c_0\\
\end{eqnarray*}

Wir l\ouml sen nun auf, indem wir die Kosten einfach "hineinschreiben"
\begin{eqnarray*}
&=& (... ((((1M + c_{n - 1})x + c_{n - 2})x + c_{n - 3})x + c_{n - 4})x + c_{n - 5}) ...)x + c_0\\
&=& (... ((((1M + 1A)x + c_{n - 2})x + c_{n - 3})x + c_{n - 4})x + c_{n - 5}) ...)x + c_0\\
&=& (... ((((1M + 1A) + 1M + 1A)x + c_{n - 3})x + c_{n - 4})x + c_{n - 5}) ...)x + c_0\\
&=& (... ((((1M + 1A) + 1M + 1A) +1M + 1A)x + c_{n - 4})x + c_{n - 5}) ...)x + c_0\\
&=& (... ((((1M + 1A) + 1M + 1A) +1M + 1A) + 1M + 1A) +1M + 1A) ...)x + c_0\\
\end{eqnarray*}
Man sieht relativ schnell am zusammenfassen, was rauskommt:
\begin{eqnarray*}
&=& (... ((((2M + 1A) + 1A) +1M + 1A) + 1M + 1A) +1M + 1A) ...)x + c_0\\
&=& (... (((2M + 2A) +1M + 1A) + 1M + 1A) +1M + 1A) ...)x + c_0\\
&=& (... (((3M + 2A) + 1A) + 1M + 1A) +1M + 1A) ...)x + c_0\\
&=& (... ((3M + 3A) + 1M + 1A) +1M + 1A) ...)x + c_0\\
&=& (... (4M + 4A) +1M + 1A) ...)x + c_0\\
&=& (nM + nA)\\
\end{eqnarray*}


Gesamt im Einheitskostenma\ss:\\
\begin{equation*}
= (nM + nA) = n + n = 2n\\
\end{equation*}
  
\end{document}

