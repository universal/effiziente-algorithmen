\documentclass[12pt,a4paper]{scrartcl}
\usepackage{ifpdf}
\usepackage[utf8]{inputenc}
\usepackage[ngerman]{babel}
\usepackage{amsmath}
\usepackage{amssymb}
\usepackage{fancyhdr}
\usepackage{listings}
\usepackage{color}
\usepackage{pdfpages}
\usepackage{stmaryrd}
\lstset{% general command to set parameter(s) 
basicstyle=\small	% print whole listing small
}

\renewcommand{\author}[1]{\newcommand{\printAuthor}{#1}}
\renewcommand{\title}[1]{\newcommand{\printTitle}{#1}}
\title{Serie 01}
\author{Johannes Hedtrich, Nikolaus Majewski}
\date{\today}

\pagestyle{fancy}
\fancyhead{}
\fancyfoot{}
\fancyhead[LO,LE]{\printAuthor}
\fancyhead[RO,RE]{\printTitle}
\fancyfoot[CO,CE]{\thepage}
 
\title{Serie 07}

\begin{document} 

  \section*{Aufgabe 7.3 Matrizen}
  \textbf{Voraussetzung: } Seien $A,B,C$ $n \times n$- Matrizen. Für $k \leq n$ definieren wir die obere linke Teilmatrix als $(a_{ij})_{1 \leq i,j \leq k}$ und die untere rechte Teilmatrix als $(a_{ij})_{n-k+1 \leq i,j \leq n}$.  
  \noindent
  \textbf{Behauptung: } 
  \begin{enumerate}
    \item $(A^TB^T) = (BA)^T$
    \item Sind $A$ und $B$ symmetrisch, so auch $A^{-1}$ (falls definiert) und $A+B$
    \item Sind $A$ und $B$ positiv definit, so auch $A^{-1}$ und $A+B$
    \item $A^TA$ ist symmetrisch und positiv definit
    \item Ist $A$ positiv definit, so auch alle oberen linken und unteren rechten Teilmatrizen von A.
  \end{enumerate}

  \noindent
  \textbf{Beweis: }
  
  \subsubsection*{$(A^TB^T) = (BA)^T$}
  
  Seien $1 \leq i,j \leq n$. Dann ist $(A^TB^T)_{ij} = \sum^{n}_{k=1} A^T_{ik} B^T_{kj} = \sum^{n}_{k=1} A_{ki} \cdot B_{jk}  = \sum^{n}_{k=1} B_{jk} \cdot A_{ki} = (BA)_{ji} = (BA)^T_{ij}$.

  \subsubsection*{Ist $A$ symmetrisch, so auch $A^{-1}$}
  Für $1 \leq i,j \leq n$ ist $A_{ij} =  A_{ji}$. Zudem ist $AA^{-1} = I$, also $I_{ij} = I_{ji}$. 
  
  $I_{ij}$ ergibt sich aus $\sum^{n}_{k=1} A_{ik} \cdot A^{-1}_{kj} = \sum^{n}_{k=1} A_{ki} \cdot A^{-1}_{kj}$.
  
  $I_{ji}$ ergibt sich aus $\sum^{n}_{k=1} A_{jk} \cdot A^{-1}_{ki} = \sum^{n}_{k=1} A_{kj} \cdot A^{-1}_{ki}$.
  
  \subsubsection*{Sind $A$ und $B$ symmetrisch, so auch $A+B$}
  
  Sei $C = A + B$. Für $1 \leq i,j \leq n$ ist $C_{ij} = A_{ij} + B_{ij} = A_{ji} + B_{ji}$ = $C_{ji}$.
  
  
  
\end{document}