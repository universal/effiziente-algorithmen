\documentclass[12pt,a4paper]{scrartcl}
\usepackage{ifpdf}
\usepackage[utf8]{inputenc}
\usepackage[ngerman]{babel}
\usepackage{amsmath}
\usepackage{amssymb}
\usepackage{fancyhdr}
\usepackage{listings}
\usepackage{color}
\usepackage{pdfpages}
\usepackage{stmaryrd}
\lstset{% general command to set parameter(s) 
basicstyle=\small	% print whole listing small
}

\renewcommand{\author}[1]{\newcommand{\printAuthor}{#1}}
\renewcommand{\title}[1]{\newcommand{\printTitle}{#1}}
\title{Serie 01}
\author{Johannes Hedtrich, Nikolaus Majewski}
\date{\today}

\pagestyle{fancy}
\fancyhead{}
\fancyfoot{}
\fancyhead[LO,LE]{\printAuthor}
\fancyhead[RO,RE]{\printTitle}
\fancyfoot[CO,CE]{\thepage}
 
\usepackage{program}
\title{Serie 09}

\begin{document} 

  \section*{Aufgabe 11.3 2Sat}
  \textbf{Voraussetzung: } Sei $\mathcal{M}$ die 2-Klauselmenge einer Formel in KNF in der jede Klausel nur zwei Literale enthält.

  \noindent
  \textbf{Algorithmus:}
  \begin{program}
  \mbox{DP-Sat(M)}
  \text{Vorbedingung: M Klauselmenge ohne tautologischen Klauseln}
  \WHILE M \neq \emptyset \text{ und } \emptyset \notin M%
  \END
  \BEGIN
    \text{wähle} X_i \in |vars|(M) \neq \emptyset
    M' = M \cup \{K \cup K' : K \dot{\cup} \{X_i\} \in M \text{und} K' \dot{\cup} \{\neg X_i\} \in M\}
    M'' = \{ K \in M': X_i \notin |vars|(K)\}
    M = \{ K \in M'' : K \text{nicht tautologisch} \}
  \END
  \IF \emptyset \in M \do
    \text{return unerfüllbar}
  \END
  \IF \emptyset \notin M \do
    \text{return erfüllbar}
  \END
  \end{program}   

  
  \noindent
  \textbf{Behauptung: } Für 2-Klauselmengen ist die Laufzeit des DP-Algorithmus polynomiell in der Anzahl der Variablen. D.h., das Erfüllbarkeitsproblem ist für 2-Klauselmengen in Polynomzeit lösbar.
  
  \noindent
  \textbf{Beweis: }
  
  Sei $m = vars(\mathcal{M})$. Die Menge $m$ wird in jedem Schleifendurchlauf um Eins verringert, allerdings kann sich die Anzahl der Klauseln(und deren Länge) durch Ausführen eines Schleifendurchlaufs drastisch(quadratisch) erhöhen. Somit wird die Schleife maximal $m$-mal durchlaufen. Eine einmalige Ausführung des Schleifendurchlaufs benötigt eine Laufzeit, die grob geschätzt polynomiell in $N = \sum_{K \in \mathcal{M}} \origbar K \origbar$. Der Algorithmus verändert die Eigenschaft, dass $\mathcal{M}$ eine 2-Klauselmenge ist nicht. Somit ist $N$ beschränkt durch die Anzahl der 2-Klauseln, deren Variablen sich unter den Variablen in $vars(\mathcal{M})$ befinden. Es gibt eine 0-Klausel dieser Art, $2m$ 1-Klauseln dieser Art und $2m \times (2m - 1)$ 2-Klauseln dieser Art, die keine 1-Klauseln sind. Insgesamt gilt also $N \leq 1+ 2m + 2m(2m - 1)$, wobei letzteres $4m^2 + 1$ ist. Hieraus folgt die Behauptung.
 
  \section*{Aufgabe 11.4 Monotone SAT}
  \textbf{Voraussetzung: } SAT ist NP-vollständig. Sei $F$ eine Formel.
  
  \noindent
  \textbf{Behauptung: } SAT ist auch dann noch NP-vollständig, wenn in jeder Klausel nur negierte oder nur unnegierte Variablen vorkommen.
  
  \noindent
  \textbf{Beweis: }
  
  Annahme: Es existiert ein Algorithmus $A \in P$ für Monotone SAT.
  
  Benenne jede negierte Variable in $F$ von $X_i$ in $NX_i$ um. Füge $F$ zwei weitere Klauseln hinzu: $\{ X_i \vee NX_i \}$ und $\{ \neg X_i \vee \neg NX_i \}$. Somit kommen in jeder Klausel der Formel nur negierte oder nur unnegierte Variablen vor. Somit wäre auch allgemeines SAT  $\in P$. $\lightning$
  
  
\end{document}